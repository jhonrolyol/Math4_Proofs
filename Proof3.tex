% Preamble
\documentclass[10pt,a4paper, openany ]{book}
% Packages
\usepackage[utf8]{inputenc}
\usepackage[spanish]{babel}
\usepackage{latexsym,amsmath,amssymb,amsfonts,amsthm} 
\newtheorem{defi}{Definición}[section]
\newtheorem{lema}{Lema}
\newtheorem{coro}{Corolario}[section]
\newtheorem{teo}{Teorema}[section]
\usepackage{graphicx}
\usepackage{ragged2e}
\usepackage{xcolor}
\usepackage{setspace}
\providecommand{\abs}[1]{\lvert#1\rvert}
\usepackage{mathrsfs}
%\usepackage{background}
%\backgroundsetup{placement = center,
% angle=0,scale=1.1,contents= {\includegraphics[scale =4]{oro.jpg}}, opacity }
\definecolor{verdeclaro}{rgb}{0.0, 1.0, 0.5}
\definecolor{rojo}{rgb}{0.89, 0.04, 0.36}
\definecolor{violeta}{rgb}{0.89, 0.04, 0.36}
\usepackage{mathtools}
\DeclarePairedDelimiter\ceil{\lceil}{\rceil}
\DeclarePairedDelimiter\floor{\lfloor}{\rfloor}
\usepackage{lipsum}
\usepackage{}
\usepackage{xcolor}
\usepackage[T1]{fontenc}
\usepackage{mathptmx}
\usepackage{amsmath,amssymb,amsfonts}
\usepackage[top=1in,bottom=1in,left=3.2cm,right=2.6cm]{geometry}
\usepackage{graphicx}\graphicspath{{graphics/}}
\usepackage{subcaption}
\usepackage{setspace}
\usepackage[square,sort&compress]{natbib}
\usepackage{xcolor}
\usepackage{booktabs}
\usepackage{makeidx}\makeindex
\usepackage[nottoc]{tocbibind}
\usepackage{emptypage}
\usepackage[grey]{quotchap}
\usepackage{hyperref}
% \pagestyle{empty}
\definecolor{letra}{RGB}{13,17,14}
\definecolor{title}{rgb}{1.0,0.03,0.0}
\definecolor{pagecolor}{RGB}{13,17,14}

% Document
\begin{document}
  \pagecolor{pagecolor}
  %%%%%%%%%%%%%%%%%%%%%%%% Proof 3 %%%%%%%%%%%%%%%%%%%%%
  \color{verdeclaro}
  \newpage
  \Large
  \begin{center}
        \textbf{PROBLEMA 3}
    \end{center}
    En el modelo
    \begin{equation*}
        Y_{i} = \beta_{1}X_{1i}+\beta_{2}X_{2i}+\mu_{i} \, \, , \, \, \, \textup{i} = 1,2,\dots ,n
    \end{equation*}
    Sabemos que
    \begin{equation*}
        X^{t}X = \begin{pmatrix}
    10 & 0 \\
      0 & 20 
    \end{pmatrix}
    \end{equation*}
    \begin{equation*}
        X^{t}Y = \begin{pmatrix}
    3 \\
      5  
    \end{pmatrix}
    \end{equation*}
    Determine la estimación de los coeficientes del modelo propuesto si $\beta_{1} = \beta_{2}$
    \begin{center}
        \textbf{Solución}
    \end{center}

    \noindent   Dado que tenemos el siguiente modelo de tres variables 
    \[
    Y_{i} = \beta_{1}X_{1i}+\beta_{2}X_{2i}+\mu_{i}
    \]
    Ahora, aplicando el operador de esperanza matemática se tiene la siguiente función de regresión poblacional
    \[
    E(Y_{i}) = \beta_{1}X_{1i}+\beta_{2}X_{2i} \, \, \,\, , \, \, \, E(\mu_{i}) = 0
    \]
    Condición de $\beta_{1} = \beta_{2} = \beta^{*}$
    \[
    E(Y_{i}) = \beta^{*}\left(X_{1i}+X_{2i}\right)
    \]
    Dado que el propósito es estimar los parámetros de la función de regresión poblacional, entonces se considera la siguiente estimación 
    \[
    \hat{Y}_{i} = \hat{\beta}_{1}X_{1i}+\hat{\beta}_{2}X_{2i} 
    \]
    Condición de $\hat{\beta}_{1} =\hat{\beta}_{2} = \hat{\beta}^{*}$\\\\
    \[
    \hat{Y}_{i} =\hat{\beta}^{*} \left(X_{1i}+X_{2i}\right) 
    \]
    
    
    \noindent Como la variable observada no siempre coincide con la variable estimada existe la posibilidad de que se generen errores muestrales de forma que por definición se tiene:
    \[
    e_{i} = Y_{i}-\hat{Y}_{1} = Y_{i}-\hat{\beta}^{*} \left(X_{1i}+X_{2i}\right)  
    \]

  \noindent entonces, para poder obtener el estimador 
    \[
      Min \, \, \,  \sum\limits_{i=1}^{n} e_{i}^{2} = \sum\limits_{i=1}^{n} \left(Y_{i}-\hat{Y}_{i}\right)^{2}  = \sum\limits_{i=1}^{n} \left(Y_{i}- \hat{\beta}^{*} \left(X_{1i}+X_{2i}\right)    \right)^{2}
    \]
  Por la condición de primer orden se tiene

  \[
  \dfrac{\partial }{\partial \hat{\beta}^{*}}\sum\limits_{i=1}^{n} e_{i}^{2} = 0 \leftrightarrow 2 \sum\limits_{i=1}^{n} \left(Y_{i}- \hat{\beta}^{*} \left(X_{1i}+X_{2i}\right)    \right)^{2}\left(-\left(X_{1i}+X_{2i}\right) \right) = 0 
  \]
  \[
  \hat{\beta}^{*} = \dfrac{X_{11}Y_{1}+X_{12}Y_{2}+X_{21}Y_{1}+X_{22}Y_{2}}{\left(X_{11}+X_{21}\right)^{2} + \left(X_{12}+X_{22}\right)^{2}}\, \, \, \, \,\, \, \, \, \, \bigstar
  \]
  Del modelo 
  \[
  Y_{i} = \beta_{1}X_{1i}+\beta_{2}X_{2i}+\mu_{i}
  \]
  se tiene para $i = 1,2 $
  \begin{align*}
    Y_{1} &= \beta_{1}X_{11}+\beta_{2}X_{21}+\mu_{1}\\
    Y_{2} &= \beta_{1}X_{12}+\beta_{2}X_{22}+\mu_{2}
  \end{align*}
  Por tanto, en forma matricial se tiene 

  \[
  \left(\begin{array}{cccc}
      Y_{1}  \\
      Y_{2} 
  \end{array}\right) =\left(\begin{array}{cccc}
      X_{11}  & X_{12} \\
      X_{21}  & X_{22} 
  \end{array}\right) \left(\begin{array}{cccc}
      \beta_{1} \\
      \beta_{2} 
  \end{array}\right) +  \left(\begin{array}{cccc}
      \mu_{1} \\
      \mu_{2} 
  \end{array}\right)
  \]
  Del cual 
  \[
  Y = \left( \begin{array}{cccc}
        Y_{1} \\
        Y_{2}
  \end{array}  \right) \, \, \, \, ,  \, \, \, \, X = \left( \begin{array}{cccc}
      X_{11}  & X_{12} \\
      X_{21}  & X_{22} 
  \end{array}\right) 
  \]
  Por consiguiente, haciendo algunos cálculos básicos se tiene 
  \[
  X^{'}X = \left( \begin{array}{cccc}
        X_{11}^{2}+X_{12}^{2} & X_{11}X_{21}+X_{12}X_{22} \\
        X_{21}X_{11}+X_{22}X_{12} & X_{21}^{2}+X_{22}^{2} 
  \end{array}  \right)      \, \, \, \, , \, \, \, X^{'}Y = \left( \begin{array}{cccc}
        X_{11}Y_{1} + X_{12}Y_{2} \\
        X_{21}Y_{1} + X_{22}Y_{2}
  \end{array}  \right) 
  \]
  Y dado que tenemos como dato
  \[
    X^{'}X = \left (\begin{array}{cccc}
        10 & 0 \\
        0  & 20
    \end{array}\right) \, \, \textup{,} \, \, X^{'}Y = \left(\begin{array}{cccc}
          3 \\
          5 
    \end{array} \right)
    \]
  se pueden igualar los valores de las matrices
  \[
  X_{11}^{2}+X_{12}^{2} = 10 \, \, , \, \, X_{21}^{2}+X_{22}^{2} = 20 \, \, , \, \,X_{11}X_{21}+X_{12}X_{22} = 0  \, \, , \, \, X_{21}X_{11}+X_{22}X_{12}  = 0 
  \]
  \[
  X_{11}Y_{1} + X_{12}Y_{2}  = 3  \, \, , \, \,  X_{21}Y_{1} + X_{22}Y_{2} = 5
  \]
  de la igualdad de matrices, se tiene 
  \[
  X_{11}X_{21} = - X_{12}X_{22}
  \]
  Por lo tanto
  \[
  \hat{\beta}^{*} = \dfrac{X_{11}Y_{1}+X_{12}Y_{2}+X_{21}Y_{1}+X_{22}Y_{2}}{\left(X_{11}+X_{21}\right)^{2} + \left(X_{12}+X_{22}\right)^{2}} = \dfrac{3+5}{30} = \dfrac{4}{15} = 0.2666\dots \, \, \, \, \, \, \, \, \, \, \, \, \, \, \, \, \, \,    \blacksquare
  \]

\end{document}
\pagecolor{pagecolor}