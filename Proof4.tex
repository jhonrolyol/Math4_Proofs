% Preamble
\documentclass[10pt,a4paper, openany ]{book}
% Packages
\usepackage[utf8]{inputenc}
\usepackage[spanish]{babel}
\usepackage{latexsym,amsmath,amssymb,amsfonts,amsthm} 
\newtheorem{defi}{Definición}[section]
\newtheorem{lema}{Lema}
\newtheorem{coro}{Corolario}[section]
\newtheorem{teo}{Teorema}[section]
\usepackage{graphicx}
\usepackage{ragged2e}
\usepackage{xcolor}
\usepackage{setspace}
\providecommand{\abs}[1]{\lvert#1\rvert}
\usepackage{mathrsfs}
%\usepackage{background}
%\backgroundsetup{placement = center,
% angle=0,scale=1.1,contents= {\includegraphics[scale =4]{oro.jpg}}, opacity }
\definecolor{verdeclaro}{rgb}{0.0, 1.0, 0.5}
\definecolor{rojo}{rgb}{0.89, 0.04, 0.36}
\definecolor{violeta}{rgb}{0.89, 0.04, 0.36}
\usepackage{mathtools}
\DeclarePairedDelimiter\ceil{\lceil}{\rceil}
\DeclarePairedDelimiter\floor{\lfloor}{\rfloor}
\usepackage{lipsum}
\usepackage{}
\usepackage{xcolor}
\usepackage[T1]{fontenc}
\usepackage{mathptmx}
\usepackage{amsmath,amssymb,amsfonts}
\usepackage[top=1in,bottom=1in,left=3.2cm,right=2.6cm]{geometry}
\usepackage{graphicx}\graphicspath{{graphics/}}
\usepackage{subcaption}
\usepackage{setspace}
\usepackage[square,sort&compress]{natbib}
\usepackage{xcolor}
\usepackage{booktabs}
\usepackage{makeidx}\makeindex
\usepackage[nottoc]{tocbibind}
\usepackage{emptypage}
\usepackage[grey]{quotchap}
\usepackage{hyperref}
% \pagestyle{empty}
\definecolor{letra}{RGB}{13,17,14}
\definecolor{title}{rgb}{1.0,0.03,0.0}
\definecolor{pagecolor}{RGB}{13,17,14}

% Document
\begin{document}
  %%%%%%%%%%%%%%%%%%%%%%%% Proof 4 %%%%%%%%%%%%%%%%%%%%%
  \pagecolor{pagecolor}
  \color{verdeclaro}
  \newpage
  \Large
  \newpage
  \begin{center}
        \textbf{PROBLEMA 4}
    \end{center}
    Se ha estimado el modelo 
    \begin{equation*}
        Y_{t} = 4+3.8D_{1t}+12D_{2t}-4D_{3t}+0.5X_{t}+e_{t}
    \end{equation*}
    Para el período comprendido entre el primer trimestre de 1980 y el cuarto trimestre de 2001, donde las variables $D_{it}$, i=1,2,3 son variables ficticias para recoger el comportamiento estacional, es decir $D_{it}=1$ para el trimestre  i-ésimo y $D_{it}=0$ para el resto. Determine el término constante para el primer trimestre de 1995.
    \begin{center}
        \textbf{Solución}
    \end{center}
    
    \noindent Dada que ya tenemos el modelo estimado 

    \begin{equation*}
      Y_{t} = 4+3.8D_{1t}+12D_{2t}-4D_{3t}+0.5X_{t}+e_{t}
    \end{equation*}
    
    \begin{equation*}
      Y_{t} = \hat{Y}_{t} +e_{t}  \, \, \, \, \, \,  \textup{o} \, \, \, \, \, \, \, \,  e_{t} = Y_{t}-\hat{Y}_{t}
    \end{equation*}

    \noindent Entonces 

    \begin{equation*}
      \hat{Y}_{t} = 4+3.8D_{1t}+12D_{2t}-4D_{3t}+0.5X_{t}
    \end{equation*}

    \noindent Como dato tenemos  

    \begin{itemize}
        \item $D_{1(1995)} = 1$
        \item $D_{2(1995)} = 0$
        \item $D_{3(1995)} = 0$
    \end{itemize}


    \noindent Por consiguiente, el modelo estimado para el año de 1995 es :

    \begin{equation*}
      \hat{Y}_{1995} = 4+3.8D_{1(1995)}+12D_{2(1995)}-4D_{3(1995)}+0.5X_{(1995)}
    \end{equation*}
    
    \noindent Y si evaluamos el modelo para el primer trimestre del año 1995, obtenemos

    \begin{equation*}
      \hat{Y}_{1995} = 4 + 3.8(1) + 12(0) - 4(0) + 0.5X_{(1995)}
    \end{equation*}

  
    \begin{equation*}
      \hat{Y}_{1995} = 7.8 + 0.5X_{(1995)}
    \end{equation*}

    \noindent Por lo tanto, el termino constante para el primer trimestre de 1995 es $7.8$.
    

\end{document}
\pagecolor{pagecolor}