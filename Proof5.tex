% Preamble
\documentclass[10pt,a4paper, openany ]{book}
% Packages
\usepackage[utf8]{inputenc}
\usepackage[spanish]{babel}
\usepackage{latexsym,amsmath,amssymb,amsfonts,amsthm} 
\newtheorem{defi}{Definición}[section]
\newtheorem{lema}{Lema}
\newtheorem{coro}{Corolario}[section]
\newtheorem{teo}{Teorema}[section]
\usepackage{graphicx}
\usepackage{ragged2e}
\usepackage{xcolor}
\usepackage{setspace}
\providecommand{\abs}[1]{\lvert#1\rvert}
\usepackage{mathrsfs}
%\usepackage{background}
%\backgroundsetup{placement = center,
% angle=0,scale=1.1,contents= {\includegraphics[scale =4]{oro.jpg}}, opacity }
\definecolor{verdeclaro}{rgb}{0.0, 1.0, 0.5}
\definecolor{rojo}{rgb}{0.89, 0.04, 0.36}
\definecolor{violeta}{rgb}{0.89, 0.04, 0.36}
\usepackage{mathtools}
\DeclarePairedDelimiter\ceil{\lceil}{\rceil}
\DeclarePairedDelimiter\floor{\lfloor}{\rfloor}
\usepackage{lipsum}
\usepackage{}
\usepackage{xcolor}
\usepackage[T1]{fontenc}
\usepackage{mathptmx}
\usepackage{amsmath,amssymb,amsfonts}
\usepackage[top=1in,bottom=1in,left=3.2cm,right=2.6cm]{geometry}
\usepackage{graphicx}\graphicspath{{graphics/}}
\usepackage{subcaption}
\usepackage{setspace}
\usepackage[square,sort&compress]{natbib}
\usepackage{xcolor}
\usepackage{booktabs}
\usepackage{makeidx}\makeindex
\usepackage[nottoc]{tocbibind}
\usepackage{emptypage}
\usepackage[grey]{quotchap}
\usepackage{hyperref}
% \pagestyle{empty}
\definecolor{letra}{RGB}{13,17,14}
\definecolor{title}{rgb}{1.0,0.03,0.0}
\definecolor{pagecolor}{RGB}{13,17,14}

% Document
\begin{document}
  %%%%%%%%%%%%%%%%%%%%%%%% Proof 5 %%%%%%%%%%%%%%%%%%%%%
  \pagecolor{pagecolor}
  \color{verdeclaro}
  \newpage
  \Large
  \newpage
    \begin{center}
        \textbf{PROBLEMA 5}
    \end{center}
    Se a estimado el siguiente modelo con 100 observaciones 
    \begin{equation*}
        Y_{t} = 2 +7D_{1t}+4.37X_{t}+e_{t}
    \end{equation*}
    \begin{equation*}
        (0.5)\, \, (0.5) \, \, (1.21)
    \end{equation*}
    Donde los valores entre paréntesis son los errores estandar de las estimaciones de los parámetros y $D_{1t}=1$ para las observaciones 1 $\leq$t $\leq$50 y $D_{1t}=0$ para las observaciones 51 $\leq$ t$\leq$ 100.
    Contraste la hipótesis de que los términos constantes para ambas submuestras ($t_{1}$=1,2,3,$\dots$,50 y $t_{2}$=51,52,53,$\dots$,100) son iguales.
    \begin{center}
        \textbf{Solución}
    \end{center}

\end{document}
\pagecolor{pagecolor}